\documentclass[11pt]{article}
\usepackage{amsmath, amssymb, amscd, amsthm, amsfonts}
\usepackage{graphicx}
\usepackage{hyperref}
\usepackage{float}

\oddsidemargin 0pt
\evensidemargin 0pt
\marginparwidth 40pt
\marginparsep 10pt
\topmargin -20pt
\headsep 10pt
\textheight 8.7in
\textwidth 6.65in
\linespread{1.2}

\title{Teroía del gas ideal}
\author{Santiago Tabares, Andrés Muñoz y Jhoan Eusse}
\date{}

\begin{document}

\maketitle

\section{Descripición del gas ideal}

El gas ideal es un modelo simple de la mécanica estadística que exitosamente explica muchas propiedades de los gases reales, tales como su capacidad calorífica. 

El gas consiste de N partículas clásicas con posición $\vec{x_{i}}$ y velocidad $\vec{v_i}$, contenidas en una caja de lado L. Cada partícula es una "esfera rígida" de radio $r_{i}$ y masa $m_{i}$. Las partículas no interaccionan por medio de un potencial, sino que únicamente colisionan  elásticamente una con otra y se reflejan(especularmente) con las paredes de la caja que las contienen. Como todas las interacciones son elásticas, la energía total del sistema 

\begin{equation}
    E = \sum_{i} \dfrac{1}{2}m_{i}|\vec{v_{i}}|^{2}
    \end{equation}
se conserva.

\subsection{Ecuación de movimiento}

Ahora, vamos a describir la evolución en el tiempo de estas colisiones de partículas clásicas. Cuando una partícula está viajando libremente a través del espacio, se mueve a una velocidad fija con una trayectoria dada por 

\begin{equation}
    \vec{x_{i}(t + \Delta t)} = \vec{x_{i}}(t) + \vec{v_{i}}\Delta t 
\end{equation}

Una partícula es reflejada por una pared, con un vector unitario normal a la superficie $\hat{n}$ en una posición $w$ a lo largo de la coordenada $\hat{n}$, si la partícula toca la pared, que es, si 

\begin{equation}
    \vec{x_{i}} - r_{i}\hat{n} = w
\end{equation}

Ya que solo estamos considerando reflexiones especulares, la componente de velocidad normal a la pared is invertida, significando que su velocidad cambie como

\begin{equation}
    \vec{v_{i}} \rightarrow\vec{v_{i}} - 2(\vec{v_{i}}\cdot\hat{n})\hat{n}
\end{equation}

Una colisión entre 2 partículas $i$ y $j$ ocurre cuando 
\begin{equation}
    |\vec{x_{i}} - \vec{x_{j}}| = r_{i} + r_{j}

\end{equation}

o cuando las partículas se tocan. DUrante la colisión, hay un intercambio de momento $\vec{q}$ entre las partículas:

\begin{equation} 
\begin{split}
\vec{v_{i}} & \rightarrow \vec{v_{i}} + \vec{q}/m_{i}  \\
 \vec{v_{j}} & \rightarrow \vec{v_{j}} - \vec{q}/m_{j}
\end{split}
\end{equation}

Para fijar $\vec{q}$, aplicamos 2 ligaduras. Ya que solo consideramos colisiones elásticas donde la energía cinética se conserva, se debe satisfacer 

\begin{equation}
    \dfrac{1}{2}m_{i}v_{i}^{2}+\dfrac{1}{2}m_{j}v_{j}^{2} = \dfrac{1}{2}m_{i}|\vec{v_{i}} + \vec{q}/m_{i}|^{2} + \dfrac{1}{2}m_{j}|\vec{v_{j}} - \vec{q}/m_{j}|^{2} 
\end{equation}

y ya que solo consideramos "colisiones especulares", "\vec{q}" debe orientarse normal al plano de reflexión, esto es,

\begin{equation}
    \vec{q} = q\hat{r_{ij}},
    
\end{equation}

donde $\hat{r_{ij}} = (\vec{r_{i}} - \vec{r_{j}})/|\vec{r_{i}} - \vec{r_{j}}|$ es un vector unitario relativo entre el centro de las partículas. Resolviendo estas ecuaciones para q obtenemos que el momento transferido ecuando dos partículas colisionan es

\begin{equation}
    \vec{q} = -2\dfrac{m_{i}m_{j}}{m_{i} + m_{j}}[(\vec{v_{i}} - \vec{v_{j}})\cdot\hat{r_{ij}}]\hat{r_{ij}}
\end{equation}

\subsection{Distribución de Maxwell-Boltzmann}

Curiosamente, no importa como las partículas del gas se encuentren en sus condiciones iniciales siempre van a tender a un estado de equilibrio. La tendencia experimental de las partículas a acercarse a un cierto equilibrio de velocidades se apoya sobre una explicación teórica simple.

Partamos de un argumento clásico de la maximización de la entropía para derivar la distribución de velocidades de Maxwell.Boltzmann. 

Consideremos un sistema de N partículas conpartiendo una energía total $E$. Supongamos que la energías posibles están dadas por $\epsilon_{i}$. Deseamos saber la distribución de equilibrio de energías, es decir, cuántas partículas tienen energía $\epsilon_{i}$.

Sabemos que para un gas ideal, el número de miscroestados está dado por 

\begin{equation}
    W = \dfrac{N!}{\prod_{i}n_{i}!}
\end{equation}

donde $N!$ cuenta el número total de formas de re-asignar energías a las partículas, y $n_{i}$ corrige el problema de sobre conteo para diferentes partículas teneiendo la misma energía $\epsilon{i}$. Una forma de simplificar la expresión es tomar el logaritmo de los microestados. Podemos usar la aproximación de Sterling $\log n! \approx n\log n- n$ para el límite de $n$ grandes:

\begin{equation}
    \log W \approx N\log N - N -\sum_{i}(n_{i}\log n_{i} - n_{i}) 
\end{equation}
\begin{equation}
    = N\log N - \sum_{i}n_{i}\log n_{i}
\end{equation}

Ahora, nuestra tarea se resume en econtrar la distribución {$n_{i}$} que máximiza $\log W$, sujeta a las ligaduras 

\begin{equation*}
    \sum_{i}n_{i} = N; \sum_{i}\epsilon_{i}n_{i} = E.
\end{equation*}

INtroduciendo los multiplicadores de Lagrange $\alpha$ y $\beta$ para implementar estas ligaduras, encontramos que se debe satisfacer 

\begin{equation*}
    G({n_{i}}) = \log W + \alpha(ligadura 1) + \beta(ligadura 2)
\end{equation*}
\begin{equation}
    = N\log N - \sum_{i}\log n_{i} + \alpha(N - \sum_{i}n_{i}) + \beta(E - \sum_{i}\epsilon_{i}n_{i})
\end{equation}

Tomando la derivada de $G$ con respecto a $n_{j}$, encontramos 

\begin{equation}
    \dfrac{dG}{dn_{j}} = -\log n_{j} - 1 -\alpha - \beta\epsilon_{i}
\end{equation}
y estableciendo el críterio de la primera derivada para encontrar el máximo de $n_{j}$, llegamos al siguiente resultado 

\begin{equation}
    n_{j} = e^{1 - \alpha}e^{-\beta\epsilon_{i}} 	\vdots \box{\dfrac{n_{j}}{N} = \dfrac{1}{Z}e^{-\beta\epsilo_{i}}}
\end{equation}
donde definimos $Z = Ne^{1+\alpha} = \sum_{i}e^{-\beta\epsilon_{i}}$. La ecuación (15) es conocida como la distribución de Maxwell-Bolzmann; se dice que la distribución de {$n_{i}$} cae exponencialmente en la energía.

Para derivar la distribución de velocidades de la distribución de energía, necesitamos hacer unos pocos pasos adicionales. Ya que la energía cinética de una partícula esta dada por $\epsilon = \dfrac{1}{2}mv^{2}$, la distribución de velocidades es está por 

\begin{equation}
    P(\vec{v})d\vec{v} \propto e^{-\beta m|\vec{v}|^{2}/2}
\end{equation}

Para caso partícular en 2 dimensiones, el elemento de área es $d\vec{v} = 2\pi vdv$, así la distribución de velocidades de Maxwell-Boltzmann para partículas en una caja que describe un gas ideal, es finalmente 

\begin{equation}
    P(v)dv \propto ve^{-\beta mv^{2}/2}
\end{equation}

\cite{Chang}\cite{bannerman2014stable}

\bibliographystyle{ieeetr}

\bibliography{references}

\end{document}
